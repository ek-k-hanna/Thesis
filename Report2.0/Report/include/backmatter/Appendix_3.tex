\chapter{Complete Aggregation }
\label{appendix:aggregate_a}
To fully aggregate the signature-based set membership proofs the verification would need to be independent of the number of provers. 

Consider two set membership proofs $\Sigma_1$ and $\Sigma_2$ computed according to the algorithm \textbf{Prove} in Construction \ref{alg:ZKSM-Agg}. The proof $\Sigma_i = (V_i,a_i,D_i,z_{x_i},z_{\tau_i},z_{R_i})$ for $i=1,2$. Denote the Pedersen commitments by $C_i$ and the challenges $c_i$, for $i=1,2$.
It will be assumed, although unrealistic for non interactive constructions of set membership proof, that $c=c_1=c_2$.
%and $V_i =  g^{\frac{\tau_i}{q+x_i}}, 
%a_i = e(V_i,g)^{s_i}e(g,g)^{t_i}, \:
%D_i =  g^{s_i}h^{m_i}, \:
%z_{x_i} =  s_i- c x_i, \:
%z_{\tau_i} = t_i- c \tau_i$ and 
%$z_{R_i} =  m_i+ c R_i$, 
%where $s_i,t_i,m_i \in_R \mathds{F}$ ,  $g,h$ are group elements of the group $\mathds{G}$, $q$ is a secret key not known by any party, $y=g^q$ is the public key known to any party. The, here assumed same for both parties, challenge is denoted$c$ and is publicly known, and finally $x_i, R_i$ denoted the secret and randomness of prover $i=1,2$ 

The aim is to investigate if the two proofs can be aggregated into the aggregated proof, $\Sigma_a=(V_a,a_a,D_a,z_{x_a}, z_{\tau_a}, z_{R_a})$ , such that $a_a \overset{?}{=} e(V_a,y)^c e(V_a,g)^{-z_{x_a}}e(g,g)^{z_{\tau_a}}$ holds and that equality implies that $x_i\in\Phi$ for $i=1,2$. 

The aggregation is performed according equation \eqref{eq:naiveAgg}, using $c$ as the challenge. This given that,
\begin{align*}
& a_a = e(g,g) ^{\frac{\tau_1 s_1}{q+x_1} \frac{\tau_2 s_2}{q+x_2} +t_1+t_2} 
\\
& e(V_a,y)^ce(V_a,g)^{-z_{x_a}}e(g,g)^{z_{\tau_a}}  = \: ... \:= 
  e(g,g)^{ \frac{s_1 \tau_1}{q+x_1}+t_1} e(g,g)^{ \frac{s_2 \tau_2}{q+x_2}+t_2}   e(V_1,g)^{- z_{x_2} }   e(V_2,g)^{ - z_{x_1} }
   \\
 & \implies a_a \neq e(V_a,y)^ce(V_a,g)^{-z_{x_a}}e(g,g)^{z_{\tau_a}} 
\end{align*}
%e(V_1V_2,y)^ce(V_1V_2,g)^{-z_{x_1}- z_{x_2}}e(g,g)^{z_\tau_1+z_\tau_2} \\
%=& e(g,g)^{cq \big (  \frac{\tau_1}{q+x_1} + \frac{\tau_2}{q+x_2}\big) } e(g,g)^{ \big(-(s_1- c x_1) - (s_2 - c x_2)  \big) \big (  \frac{\tau_1}{q+x_1} + \frac{\tau_2}{q+x_2}\big) } e(g,g)^{ (t_1- c \tau_1) +(t_2- c \tau_2)} \\
 %= & e(g,g)^{ \frac{\tau_1}{q+x_1} \Big(  cq- (s_1-cx_1) -(s_2-cx_2) -c(q+x_1) \Big)+ t_1 }  e(g,g)^{ \frac{\tau_2}{q+x_2} \Big(  cq - (s_1-cx_1)- (s_2-cx_2) + c)(q+x_2) \Big)+ t_2 } \\
 %= &  e(g,g)^{ \frac{s_1 \tau_1}{q+x_1}+t_1}  e(g,g)^{ \frac{\tau_1}{q+x_1} \Big(  cq + cx_1 - c(q+x_1) - (s_2-cx_2) \Big) }  \\
% &e(g,g)^{ \frac{s_2 \tau_2}{q+x_2} + t_2  }  e(g,g)^{ \frac{\tau_2}{q+x_2} %\Big(  cq + cx_2 - c(q+x_2) - (s_1-cx_1) \Big) } \\
% = &  e(g,g)^{ \frac{s_1 \tau_1}{q+x_1}+t_1} e(g,g)^{ \frac{s_2 \tau_2}{q+x_2}+t_2}   e(g,g)^{ \frac{\tau_1}{q+x_1} \Big(  - (s_2-cx_2) \Big) }   e(g,g)^{ \frac{\tau_2}{q+x_2} \Big( - (s_2-cx_2) \Big) } \\
% = & %

Even under the assumption that $c= c_1=c_2$, the terms $e(V_1,g)^{- z_{x_2} }   e(V_2,g)^{ - z_{x_1} }$ are not cancelled. Thereby, although the aggregation is performed such that the challenges only appears as a product the completeness property does not hold.  


