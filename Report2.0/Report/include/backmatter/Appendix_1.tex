\chapter{Multiple aggregating parties construction}
\label{app:manyAggregatingParties}


\begin{comment}
Let $x=\sum_{j=0}^{l-1} x_j u^j$, where $x_j$ is an integer and $x_j\in[0,u)$, $u,l$ are integers and $j\in[0,l-1] (=\mathds{Z}_l)$. Then it holds that $x\in[0,u^l)$.
\begin{align*}
x =& \sum_{j=0}^{l-1} x_j u^j \leq \sum_{j=0}^{l-1}  (u-1)u^j = \sum_{j=0}^{l-1} u^{j+1} - \sum_{j=0}^{l-1} u^j = (u-1) \sum_{j=0}^{l-1} u^j =\\
 &(u-1) \frac{u^l-1}{u-1} = u^l-1< u^l 
\end{align*}
Hence the  statement is proved and it is trivial to see that if $j\in[0,l]$ the value of $x$ could exceed $u^l$.
\end{comment}
\begin{comment}
\begin{algorithm}
\caption{\textbf{: Client and Server Verifiable additive homomorphic secret sharing}}

\textbf{Goal:} Construct and share the sum $\sum_{i=1}^n x_i$, where $x_i$ is a secret value known by client $c_i$, where $i\in\mathcal{N}$ without any client needing to revealing their individual secret. All parties are verified to be honest in the construction.
\vspace{2pt}
\hline
\vspace{2pt}
\begin{itemize}
 \item\textbf{ShareSecret $(1^\lambda,i,x_i) \mapsto \{x_{ij}\}_{j\in\mathcal{M}}$} \\
Pick uniformly at random $\{a_i\}_{i\in\{1,..,t\}}\in_R\mathds{F}$ to be the coefficients to a $t$-degree polynomial $p_i$ on the form $p_i(X) = x_i + a_1X+...+a_tX^t$. Define  the shares as $x_{ij}=\lambda_{i,j}p_i(\theta_{ij})$ for $j\in\mathcal{M}$, the parameters $\theta_{ij}$ and Lagrange coefficients $\lambda_{ij}$ is chosen such that equation \ref{eq:pi(0)} is satisfied.
Output $\{x_{ij}\}_{j\in\mathcal{M}}$.

\item\textbf{GenereteCommitment$(1^\lambda,i,x_i) \mapsto \pi_i$ }\\
Let $\mathds{E} : x,y \to g^xh^y$ be a Pedersen commitment . Let $R_i\in\mathds{F}$ be the output of a PRF such that $R_n\in \mathds{F}$  satisfies $R_n = \phi(N)\lceil \frac{\sum_{i=1}^{n-1}R_i}{\phi(N)}\rceil- \sum_{i=1}^{n-1}R_i $. Compute and output $\pi_i = \mathds{E}(x_i,R_i)$.

\item\textbf{RangeProof $(x_i,\pi_i) \mapsto RP_i$}\\
Construct a range proof, denoted $RP_i$, for the commitment $\pi_i$ to the secret $x_i$, on the set $\Phi$ using the agorithm \textbf{Prove} in Construction \ref{alg:ZKSM-Agg}. All required  parameters and setup is assumed to be pre-shared and known by all parties.

\item\textbf{PartialEval $(j,\{x_{ij}\}_{i\in\mathcal{N}})\xrightarrow[]{}y_j$}\\
Compute and output $y_j = \sum_{i=1}^n x_{ij}$.

\item\textbf{PartialProof $(j,\{x_{ij}\}_{i\in\mathcal{N}})\xrightarrow[]{}\sigma_j$}\\
Compute and output $\sigma_j = \prod_{i=1}^n g^{x_{ij}} =  g^{\sum_{i=1}^n x_{ij}}= g^{y_j}=H(y_j)$.

\item \text{\textbf{PartialAggregate} $(g,h, \mathcal{S}\{ \textit{proof}_{SM,i}\}_{i\in\mathcal{S}} \xrightarrow[]{} \texit{proof}_{SM,a}$} \\
Given a subset of range proofs  $\{ \textit{proof}_{SM,i}\}_{i\in\mathcal{S}}$ where $\mathcal{S}\subseteq \{1,...,n\}$. Aggregate the values $\{D_i \}_{i\in\mathcal{S} } \{ z_{x_i}\}_{i\in\mathcal{S} } \{ z_{r_i}\}_{i\in\mathcal{S} } \mapsto D_a,z_{x_a},z_{R_a}$ according to equation \eqref{eq:aggDn}. Then construct and publish the aggregated range proof $\texit{proof}_{SM,a} = (\{V_i\}_{i\in\mathcal{S} },\{a_i\}_{i\in\mathcal{S} },D_a,\{z_{x_i}\}_{i\in\mathcal{S} }, z_{x_a}, \{z_{\tau_i}\}_{i\in\mathcal{S} },z_{R_a})$.


\item\textbf{FinalEval $(\{y_j\}_{j\in\mathcal{M}})\xrightarrow[]{}y$}\\
Compute and output $y = \sum_{j=1}^m y_{j}$.

\item\textbf{FinalProof $(\{\sigma_j\}_{j\in\mathcal{M}})\xrightarrow[]{}\sigma$}\\
Compute and output $\sigma = \prod_{j=1}^m \sigma_j = \prod_{j=1}^m g^{y_{j}} =  g^{\sum_{j=1}^m y_{j}}= g^{y}=H(y)$.

\item\textbf{Verify $(\{\pi_i\}_{i\in\mathcal{N}},x,y,\{RP_{a_i}\}_{i\in\mathcal{M}})\xrightarrow[]{}\{0,1\}$}\\
Compute and output $\sigma= \prod_{i=1}^n \pi_i \wedge \prod_{i=1}^n \pi_i = H(y)\wedge \{\textbf{VerifyAggregatedProof}(RP_{a_i})\}_{i\in\mathcal{M}}$. Where $\textbf{VerifyAggregatedProof}$ is the verification algorithm in Construction \ref{alg:VAHSS-HSS-RP-Agg}.
\end{itemize}
\label{alg:VAHSS-HSS-RP-Agg}
\end{algorithm}

\end{comment}

\begin{algorithm}[]
\caption{\textbf{: Aggregation of non interactive set membership proof}}
%This protocol is a modification of Construction \ref{alg:ZKSM}, such that when multiple parties wish to prove that their secret is in an allowed set, $\Phi$, to the same verifier, their proofs can be partly aggregated into one common proof to reduce the verification time. 
\textbf{Goal:}  Given the Pedersen commitments $C_i=g^{x_i} h^{R_i}$ for $i\in\mathcal{S}$. The goal is to prove that all secrets $x_i$ belongs to the set $\Phi$. Without revealing anything else about the secrets $x_i$.
\vspace{2pt} \hline \vspace{2pt}
\begin{itemize}
  \item\textbf{SetUp $(1^\lambda,\Phi)\xrightarrow[]{}(sk,pp)$}\\
 Let g be a generator of the group $\mathds{G}$ and $h$ and element in the group such that $log_g(h)$ is unknown.  
Pick uniformly at random $\chi\in_R\mathds{F}$ and put $sk=\chi$. Define $y=g^\chi$ and $A_i=g^{\frac{1}{\chi+i}} \:\forall i\in\Phi$, output $pp=(g,h,y,\{A_i\}_{i\in\Phi})$.

\item\text{\textbf{Prove} $(pp,i,C_i,\Phi)\xrightarrow[]{}}\Sigma_i$}\\
Pick uniformly at random $\tau_i\in_R\mathds{F}$, choose from the set $\{A_i\}$ the element $A_{x_i}$ and calculate $V=A_{x_i}^{\tau_i}$. Pick uniformly random three values $s_i,t_i,m_i\in_R\mathds{F}$. Put $a_i=e(V_i,g)^{-s_i}e(g,g)^{t_i}$ ($e(\cdot,\cdot)$ is a bilinear mapping as described above), $D=g^{s_i}h^{m_i}$, and $c=\text{Hash}(C_i,V_i,a_i,D_i)$. Finally compute $z_{x_i} = s_i-x_i c_i$, $z_{R_i} = m_i-R_ic_i$ and $z_{\tau_i}= t_i-\tau_i c_i$ then construct and publish $\Sigma_i = (V_i,a_i,D_i,z_{x_i},z_{\tau_i},z_{R_i})$.

\item \text{\textbf{Aggregate} $( pp,k,\{ \Sigma_i\}_{i\in\mathcal{S}_k} ) \xrightarrow[]{} \Sigma_{a_k}$} \\
Given a set of set membership proofs  $\{\Sigma_i\}_{i\in\mathcal{S}_k}$. Aggregate the values $( \{D_i \}_{i\in\mathcal{S}_k } \{ z_{x_i}\}_{i\in\mathcal{S}_k } \{ z_{r_i}\}_{i\in\mathcal{S}_k  }) \mapsto D_{a_k},z_{x_{a_k}},z_{R_{a_k}}$ according to equation \eqref{eq:aggDn}. Then construct and publish the aggregated proof $\Sigma_{a_k} = (\{V_i\}_{i\in\mathcal{S}_k },\{a_i\}_{i\in\mathcal{S}_k },D_{a_k},\{z_{x_i}\}_{i\in\mathcal{S}_k }, z_{x_{a_k}}, \{z_{\tau_i}\}_{i\in\mathcal{S}_k },z_{R_{a_k}})$.

\item \text{ \textbf{CalculateChallenges} $(\{C_i\}_{i\in\mathcal{S}}\{\Sigma_i\}_{i\in\mathcal{S}} ) \xrightarrow[]{} \{c_i\}_{i\in\mathcal{S}}$ }\\
For all $i\in\mathcal{S}$ parse the proof $\Sigma_i$, then compute the challange $c_i = Hash(C_i,V_i,a_i,D_i)$. Finally output the set of all challenges $\{c_i\}_{i\in\mathcal{S}}$. 

\item\text{\textbf{Verify} $(pp,\{\Sigma_{a_k}\}_{k\in\mathcal{K}},\{C_i\}_{i\in\mathcal{S}},\{c_i\}_{i\in\mathcal{S}}) \xrightarrow[]{} \{0,1\}$} \\
For all $k\in\mathcal{K}$ compute the product of the challenges $c_k=\prod_{i\in\mathcal{S}_k} c_i$. Check if $D_{a_k}\overset{?}{=} \big( \prod_{i\in\mathcal{S}_k}C_i\big)^{c_k}h^{z_R_{a_k}}g^{z_x_{a_k}}$ 
Then for check if $ a_i \overset{?}{=} e(V_i,y)^c_i e(V_i,g)^{-z_{x_i}}e(g,g)^{z_{\tau_i}}$ for all $i\in\mathcal{S}$. If the equalities holds the provers has convinced the verifier that $x_i\in\Phi$ for all $x_i$ such that $i\in\mathcal{S}$ return $1$ otherwise return $0$.
\end{itemize}
\label{alg:ZKSM-Agg-Many}
\end{algorithm} 
