\chapter{Aggregated Set Membership Proofs}
\label{ch:generalAgg}

%TODO fix intro
In applications where one verifier is responsible for verifying multiple provers set membership proofs the computational complexity of the verification is important to keep low. If the set membership proofs are proof of the statement in equation \eqref{eq:SM_statement} the verifier would have to verify each proof individually. This would then leads to that the runtime for the verification grows linearly in the number of provers. The linear dependence of the number of provers for the computational complexity of the verification algorithm results in that this algorithm quickly becomes a bottleneck for such an application. 

This motivates the question if an aggregation of set membership proofs can be provided to decrees the computations required to verify multiple provers simultaneously. 

An aggregated set membership proof is a zero knowledge proof of the statement,
\begin{equation}
\begin{aligned} 
\label{eq:SMagg_statement}
    \big\{(g,h\in\mathds{G},\{C_i\}_{i\in\mathcal{S}};\{x_i\}_{i\in\mathcal{S}},\{R_i\}_{i\in\mathcal{S}}\in\mathds{F}^n)\:: 
\: &C=\prod_{i\in\mathcal{S}}C_i 
\\
 &\wedge \prod_{i\in\mathcal{S}}C_i  =   g^{\sum_{i\in\mathcal{S}}x_i} h^{\sum_{i\in\mathcal{S}}R_i} 
 \\
 &\wedge x_i \in \Phi \forall i\in\mathcal{S} \big\}.
\end{aligned}
\end{equation}
The statement implies that after successfully having performed an aggregated set membership proof the verifier is convinced that or all $i\in\mathcal{S}$  the  Pedersen commitment $C_i$ is a commitment to a secret belonging to the set $\Phi$ . 


%TODO fix can this be here?  ONLY range proofs
%A remark is that to aggregate the commitments $C_i, \: i\in\mathcal{S}$ into $C = \prod_{i\in\mathcal{S}} Ci_i$. Then constructing  a range proof that the aggregated commitment $C = g^{\sum_{i=1}^n x_i}$ belongs to the range  $[n\cdot a,n\cdot b]$, does not prove that $x_i \in [a,b]$ for all $i \in\mathcal{N}$. In other word is does not satisfy the verification property in Theorem \ref{thm:VAHSS_RP_CSV}. The value $y=\sum_{i=1}^n x_i$ is publicly known so to construct a zero knowledge range proof for $y $ provides no new information and given that $y\in [n\cdot a,n\cdot b]$ does not imply $x_i\in [a,b]$ for all $i\in\mathcal{N}$. 


Further an construction of an aggregated set membership proof is a $5$-tuple of PPT algorithms as described in Definition \ref{def:GeneralAggregation}. 

\vspace{10pt}
\begin{Mydef}[\textbf{Aggregated non-interactive set membership proof}]
\label{def:GeneralAggregation}
An aggregated set membership proof proves a statement on the form \eqref{eq:SMagg_statement} and is a $5$-tuple of PPT-algorithms (\textbf{SetUp}, \textbf{Prove}, \textbf{Aggregate}, \textbf{CalculateChallanges}, \textbf{Verify}. )
\begin{itemize}

  \item\textbf{SetUp $(1^\lambda, \Phi)\xrightarrow[]{}(pp,sk)$}\\
On the input $1^\lambda$, where $\lambda$ is the security parameter , and the set $\Phi$ the algorithm outputs a secret key $sk$ and public parameters $pp$. 

\item\text{\textbf{Prove} $(pp,i,C_i,\Phi)\xrightarrow[]{}\Sigma_i}$}\\
On input the public parameters $pp$, $i\in\mathcal{S}$ denoting the index of the prover $p_i$ and a Pedersen commitment of the $p_i$'s secret $x_i$.  The algorithm outputs a polynomial time verifiable zero knowledge proof, denoted $\Sigma_i$, that the secret $x_i$ is in the set $\Phi$. 

\item \text{\textbf{Aggregate} $ (\{ pp,\Sigma_i\}_{i\in\mathcal{S}} \xrightarrow[]{} \Sigma_a}$} \\
Given a set of set membership proofs  $\{ \Sigma_{i}\}_{i\in\mathcal{S}}$ the algorithm aggregates the proofs into one zero knowledge proof of the statement in equation \eqref{eq:SMagg_statement} denoted $\Sigma_a$. 

\item \text{ \textbf{CalculateChallenges} $(\{C_i\}_{\in\mathcal{S}},\{\sigma_i\}_{i\in\mathcal{S}} ) \xrightarrow[]{} \{c_i\}_{i\in\mathcal{S}}$ }\\\
On the input $\{\Sigma_i\}_{i\in\mathcal{S}}$ the algorithm computes and outputs the challenges  $c_i = Hash(\Sigma_i)$ for all $i\in\mathcal{S}$.

\item\text{\textbf{Verify} $(pp, \Sigma_a, \{C_i\}_{i\in\mathcal{S}},\{c_i\}_{i\in\mathcal{S}}) \xrightarrow[]{} \{0,1\}$} \\
On input the aggregated set membership proof public parameters, $pp$, aggregates proof, $\Sigma_a$, and every provers Pedersen commitment, $\{C_i\}_{i\in\mathcal{S}}$, and challenge ,$\{c_i\}_{i\in\mathcal{S}}$, the algorithm outputs either $1$ or $0$. 
\end{itemize}
\end{Mydef}
\vspace{10pt}
The algorithms  (\textbf{SetUp}, \textbf{Prove}, \textbf{Aggregate}, \textbf{CalculateChallanges}, \textbf{Verify} ) should satisfy the correctness, soundness and zero knowledge requirements in Defintion \ref{def:ZKP_agg}. These requirements can be seen as a  modification of Defnition \ref{def:ZKP} to an aggregated ZKP of the statement in equation \eqref{eq:SMagg_statement}. 

\vspace{10pt}
\begin{Mydef}
\label{def:ZKP_agg}
The algorithms in Definition \ref{def:GeneralAggregation} should be such that they fulfil the completeness, soundness and zero-knowledge properties:
\begin{itemize}
\item \textbf{Completeness} Given a witness $\Sigma_i$ satisfying the instance $x_i\in\Phi$, where $C_i$ is a Pedersen commitment of$x_i$, for all $i\in\mathcal{S}$, it should hold that
\\
 \texttt{Verify}$($\texttt{Aggrgeate}$(\{$\texttt{Prove}$(pp,i,C_i,\Phi)\}_{i\in\mathcal{S}}) )= 1$. 
\item \textbf{Soundness} If for any $i\in\mathcal{S}$ the  witness $\Sigma_i$ does not satisfy the  instance $x_i\notin\mathcal{L}$, then the probability  Prob$[ $\texttt{Verify}$($\texttt{Aggrgeate}$(\{$\texttt{Prove}$(pp,i,C_i,\Phi)\}_{i\in\mathcal{S}}) ) = 1] < \varepsilon$, for a sufficiently small $\varepsilon$. 
\item  \textbf{Zero-knowledge} 
%TODO
A proof system is \textit{honest verifier zero-knowledge} if there exist a PPT algorithm \texttt{Simulator} having access to the same input as the algorithms \texttt{Verify} and \texttt{Aggregate} but not the provers input, such that output from the \textt{Simulator} and \texttt{Prove} is indistinguishable, i.e have the same distribution given that $x\in\mathcal{L}$.  
\end{itemize}
\end{Mydef}

%\subsection*{Many aggregation}
%TODO Decise if use
