\chapter{Methods}
\label{sec:Methods}
Evaluate and choose one to implement, do implementation and construct proofs

\section{Comparison of range proofs}
In this section the different constructions presented in section \ref{sec:RF_theory} will be evaluated and compared in order to decide which method is best to combine with the VAHSS scheme described in Construction\ref{alg:VAHSS-HSS} to check clients input. Each range proof constructions pros and cons will be discussed separately but tables for comparison will also be presented. Then a final comparison will be made.

The aspects that will be considered in the evaluation of the range proofs and their compatibility with the VAHSS construction is presented is the below list;
\begin{itemize}
    \item Proof size
    \item Communication complexity
    \item Flexibility of range
    \item Assumptions and requirements 
    \item Computation complexity for prover resp. verifier
\end{itemize}

Remark that all the range proof considered aim to prove that the secret in a Pedersen commitment is in an allowed range. Thus to combine any of the range proofs with the VAHSS construction, the clients needs beyond previously computed and published values also publish a Pedersen commitment. This is investigated further in section \ref{}. Another remark is that all range proofs considered have be made non interactive using the Fiat-Shamir heuristic, even if they were originally presented as interactive constructions. 

%\subsection{Square-based range proof}

The two range proofs are in some aspects fundamentally different, one uses bilinear mapping,  while bullet proofs does not. Therefore it is not straightforward to compare them in aspects of number of operations performed by prover and verifier. First the performance of the range proofs will be discussed individually then compared in terms of runtime. 

\subsection{Signature-based set membership and range proof}

First lets discuss the communication complexity and proof size starting with the signature based set membership . This construction allows for a $\mathcal{O}(1)$- size proof that a committed value belongs to a given set $\Phi$. In order to construct such a proof $n=|\Phi|$ digital signatures needs to be known by both prover and verifier, one signature for each elements in $\Phi$. This signatures are usually shared by the verifier in the Setup phase. Sharing the digital signatures of the elements in the set $\Phi$ becomes intractable when the set is large.  A large set in this context would be a set consisting of a few hundred elements since the verifier has to publish $n$ digital signatures in the SetUp phase. 

The signature based range proof reduces this to only needing to publish $u$ digital signatures to prove a commitment is in the range $[0,u^l]$ in the SetUp phase. For he rest of the proof the prover sends $l+1$ elements from the group $\mathds{G}_1$, $l$ elements from the group $\mathds{G}_T$ and $2l+1$ field elements. 
Thus the communication complexity depends on the choice of $u,l$. Asymptotic analysis gives a communication complexity $\mathcal{O}(\frac{k}{log\:k-log\:log\:k})$, where $l=\frac{k}{log\:u}$ and $u$ put to $u=\frac{k}{log\: k}$ Here $k$ satisfies $u^l \geq 2^{k-1}$.

The signature based set membership has a a constant size, given the \textit{digital signatures} of all elements in the set. In some practical applications these signatures can be assumed to be pre shared. Therefore in applications where $\Phi$ is used many times  or when $\Phi$ is a relative small, set membership is preferred.


Next consider the computational complexity.  In the set membership construction both the prover and verifier has to perform one bilinear paring and two  exponentials over the group $\mathds{G}$. While in the range proof construction the prover need to perform $l$ bilinear mappings and $5l$ exponentials to prove a secret is in the range $[0,u^l)$ and additionally $3l$ exponentials for arbitrary ranges $[a,b]$. The verifier need to ?? Discuss on meeting.
 
An advantage of the set membership construction is allows non continuous sets. An example could be that the set $\Phi$ represents all odd numbers in a certain interval and then the prover can insure the verifier that the secret is an odd number in a given range. This is an illustrative example of the flexibility of set membership proofs compared to range proofs.

%sends $XXX$. Signature is $\mathcal{O}(n)$ or using $\sigma=\sum_{k=1}^j\sigma_ju^j$ we have $\mathcal{O}(\frac{n}{log\:n-log\:log\:n})$\\
%Third party?



\subsection{Bulletproof}
The inner product argument reduces the complexity for proving the statement  in equation \eqref{eq:IPA} from being linear in the length of the vectors to logarithmic. More precisely the prover has to send $2\lceil log_2 n \rceil$ group elements and $2$ field elements to the verifier when proving the statement, thus the commutation complexity id of order $\mathcal{O}(log_2 n)$, where $n$ is the length of the vectors. 

The computational effort for the inner product argument is dominated by $8n$ respectively $4n$ group exponentiations for the prover respectively verifier. In a non-interactive construction the verifier could instead perform only one multidimensional-exponent of size $2n+ 2log_2n +1$. This leads to a significant speed up of the verification of the argument.     

Using the inner product argument to build bullet proofs result in a communication complexity of $2\lceil log_2 n \rceil +4$ group elements and $5$ field elements, where $n$ is such that a secret  is proved to be in the range $[0,2^n)$.  A remark is that in a bulletproof construction the range always has to be an exponent on $2$, if the length of the binary representation of the secret is not a two-exponent this can be solved with padding. When extending the bulletproof to prove a secret is in an arbitrary range $[a,b]$ the communication complexity is increased by an additive term of size $2$. 


%bullet is $\mathcal{O}(log\:n)$. Bulletproof not general range. No third party?
 %logarithmic size, linear prove and verification time? 
 

\subsection{Time complexity}
\begin{table}[H]
\label{tab:compare}
\caption{Communication Complexity}
\begin{tabular}{| *{3}{c|}}
			 \hline
    									&Proof Size 		&		Set up  \\ \hline		
  Signature based SM  	&   $\mathcal{O} (1) $ 					 &			 $\mathcal{O} (n)$				\\ \hline 
  Signature based RP  	&   $\mathcal{O} (l)$   	&	$\mathcal{O} (u)$ 
  	\\ \hline
  Bulletproof   				&   $ \mathcal{O} (  log_2\: n ) $  & $\mathcal{O} (1)$ 	\\
  \hline			
\end{tabular}
 \end{table}
	

\begin{table}[H]
\label{tab:compare}
\caption{Operations of proof construction}
\begin{tabular}{| *{8}{c|}}
			 \hline
    										&   \multicolumn{2}{c|}{exponentials} 	& \multicolumn{2}{c|}{Fiels Operations}    & \multicolumn{2}{c|}{Bilinear mapping }        	\\
&  Prover & Verifier & Prover & Verifier&	Prover & Verifier 						 	\\ \hline
  Signature based SM  			&   b1  &   b2  											&   c1  &   c2    & $1$							 &d						\\ \hline 
  Signature based RP  	&   b1  &   b2 							&   $5l$ &     	 & $l$	& d	 										\\ \hline
  Bulletproof   						& b1  &   b2 												&   c1 &   c2     & d					&d2								\\ \hline	
\end{tabular}
 \end{table}
	
	\begin{table}[H]
\label{tab:compare}
\caption{Operations of proof construction}
\begin{tabular}{| *{8}{c|}}
			 \hline
    										&   \multicolumn{2}{c|}{exponentials} 	& \multicolumn{2}{c|}{Fiels Operations}    & \multicolumn{2}{c|}{Bilinear mapping }        	\\
&  Prover & Verifier & Prover & Verifier&	Prover & Verifier 						 	\\ \hline
  Signature based SM  			&   b1  &   b2  											&   c1  &   c2    & $1$							 &d						\\ \hline 
  Signature based RP  	&   b1  &   b2 							&   $5l$ &     	 & $l$	& d	 										\\ \hline
  Bulletproof   						& b1  &   b2 												&   c1 &   c2     & d					&d2								\\ \hline	
\end{tabular}
 \end{table}
	

\section{Additive homomorphic secret sharing with verification of both clients and severs }
In the VAHSS Construction \ref{alg:VAHSS-HSS} the verifiability property includes verification of the servers. In this section this will be extended to also include the clients. The value $\pi_i$ published by the clients will be modified into a Pedersen commitment on the form $\pi_i = g^{x_i}h^{R_i}$, remember $\pi_i=g^{x_iR_i}$ in the original construction presented in \cite{SumItUp}. The clients will apart from the previous commitments  also construct and publish a range proof for $\pi_i$. This allows any verifier to apart from verifying the servers also verify that the secret shared by the clients is in an certain range.  

begin{algorithm}
\caption{\textbf{: Client and Server Verifiable additive homomorphic secret sharing}}
\begin{itemize}
  \item\textbf{ShareSecret $(1^\lambda,i,x_i)\xrightarrow[]{}(\pi_i,\{x_{ij}\}_{j\in\mathcal{M}})$}\\
Pick uniformly at random $\{a_i\}_{i\in\{1,..,t\}}\in\mathds{F}$ and a $t$-degree polynomial $p_i$ on the form $p_i(X) = x_i + a_1X+...+a_tX^t$. Remark that polynomial satisfies equation \eqref{eq:pi(0)} and further that $p_i(0)=x_i$  which implies that $x_i =  \sum_{j=1}^m \lambda_{ij}p_i(\theta_{ij})$. 
%Let $H:x\to g^x$
% (g generator the multiplicative group of $\mathds{F}$)
% be a collision-resistant homomorphic hash function.
Let $P : x,y \to g^xh^y$ be a Pedersen commitment function.
 Let $R_i\in\mathds{F}$ be the output of a PRF. Where it is required that  $R_n\in \mathds{F}$  satisfies
$R_n = \phi(N)\lceil \frac{\sum_{i=1}^{n-1}R_i}{\phi(N)}\rceil- \sum_{i=1}^{n-1}R_i $. Compute $\pi_i = H(x_i,R_i)$, and put $x_{ij}=\lambda_{i,j}p_i(\theta_{ij})$. \\
Construct a range proof, denoted $RP_i$, for  $\pi_i$ to the  range $[0,B]$ using Construction \ref{alg:ZKSM}, \ref{alg:ZKRP} or \ref{alg:bullet}. All required  parameters and setup is assumed to be pre-shared and known by all parties.
The algorithm published $\pi_i$ \& $RP_i$ and $x_{i,j}$ to server $j$ for $j\in\mathcal{M}$. 

\item\textbf{PartialEval $(j,\{x_{ij}\}_{i\in\mathcal{N}})\xrightarrow[]{}y_j$}\\
Compute and publish $y_j = \sum_{i=1}^n x_{ij}$.

\item\textbf{PartialProof $(j,\{x_{ij}\}_{i\in\mathcal{N}})\xrightarrow[]{}\sigma_j$}\\
Compute and publish $\sigma_j = \prod_{i=1}^n g^{x_{ij}} =  g^{\sum_{i=1}^n x_{ij}}= g^{y_j}=H(y_j)$.

\item\textbf{FinalEval $(\{y_j\}_{j\in\mathcal{M}})\xrightarrow[]{}y$}\\
Compute and output $y = \sum_{i=1}^n y_{j}$.

\item\textbf{FinalProof $(\{\sigma_j\}_{j\in\mathcal{M}})\xrightarrow[]{}\sigma$}\\
Compute and output $\sigma = \prod_{j=1}^n \sigma_j = \prod_{j=1}^m g^{y_{j}} =  g^{\sum_{j=1}^m y_{j}}= g^{y}=H(y)$.

\item\textbf{Verify $(\{\pi_i\}_{i\in\mathcal{N}},x,y)\xrightarrow[]{}\{0,1\}$}\\
Compute and output $\sigma= \prod_{i=1}^n \pi_i \wedge \prod_{i=1}^n \pi_i = H(y)\wedge \textbf{Verify}(RP_i)$. Where \textbf{Verify} is the verification step of the range proof used by the client to construct $RP_i$.
\end{itemize}
\label{alg:VAHSS-HSS-RP}
\end{algorithm}

\begin{thm}
The client and server verifiable AHSS presented in Construction \ref{alg:VAHSS-HSS-RP} satisfies the same correctness, security and requirements as Construction \ref{alg:VAHSS-HSS} as well as the verifiability requirements: 
\begin{itemize}
 \item \textbf{Verifiability Servers}  Let $\mathcal{A}$ denote any PPT  adversary and $T$ denote the set of corrupted servers with $T\leq m$. Note that if $|T|=m$, the verifiability property holds but not the security property. The verifiability property requires that any $\mathcal{A}$ who can modify the input shares to all servers $s_j\in T$ can cause a wrong value to be excepted as $y=f(x_1,...,x_n)$ with negligible probability.  
 \item  \textbf{Verifiability Clients} 
\end{itemize} 
\end{thm}
\begin{proof}
The proof of security is the same as in \cite{SumItUp} since the pedersen commitment is perfectly hiding. For proving the correctness it is sufficient to show that $\sigma= \prod_{i=1}^n \pi_i \:\bigwedge\: \prod_{i=1}^n \pi_i = \mathcal{H}(y)$. Both $y$ and $\sigma$ are the same as in construction as in \cite{VAHSS}. Hence by construction:
\begin{align}
    \label{eq:y=sum(x_ij)}
    y = \sum_{j=1}^m y_j= \sum_{j=1}^m \sum_{i=1}^n \lambda_{ij}p_i(\theta_{ij}) = \sum_{i=1}^n \overbrace{ \Big (\sum_{j=1}^m \lambda_{ij}p_i(\theta_{ij}) \Big)}^{ p_i(0)} = \sum_{i=1}^n p_i(0) = \sum_{i=1}^n x_i,
\end{align}
and for $\sigma$ it holds that:
\begin{align*}
    \sigma = \prod_{j=1}^m \sigma_j = \prod_{j=1}^m g^{y_j} = g^{\sum_{j=1}^my_j} =g^y = \mathcal{H}(y)
\end{align*}
For the $\tau_i$, whose construction has been modified compared to \cite{VAHSS} we have:
\begin{align*}
    &\prod_{i=1}^n \tau_i = \prod_{i=1}^n \mathds{E}(x_i,R_i)= \prod_{i=1}^n g^{x_i}h^{R_i} = g^{\sum_{i=1}^n x_i } h^{\sum_{i=1}^n R_i} \overset{\eqref{eq:y=sum(x_ij)}}{=} g^y h^{\sum_{i=1}^{n-1} R_i+R_n} = \\ 
    &= g^y h^{ \phi(N)\big\lceil \frac{\sum_{i=1}^{n-1}R_i}{\phi(N) }\big\rceil}  \overset{*}{=} g^y = \mathcal{H}(y) \quad \textit{*- since $h$ is co-prime to $N$.}
\end{align*}

The proof of \textit{\textbf{Verifiability Severs}} is the same as in \cite{SumItUp} and the proof of \textit{\textbf{Verifiability Clients}} follows from the propertied of range proof.
\end{proof}

\section{Implementation}