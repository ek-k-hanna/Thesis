\chapter{Methods}
\label{sec:Methods}

Evaluate and choose one to implement, do implementation and construct proofs
\section{Comparison of constructions for verifying clients input}
In this section the different constructions presented in section \ref{sec:RF_theory} will be evaluated and compared in order to decide which method is best to combine with the VHASS scheme described in Construction\ref{alg:VHASS-HSS} to check clients input. Each range proof constructions pros and cons will be discussed separately but tables for comparison will also be presented. Then a final comparison will be made.

The aspects that will be considered in the evaluation of the range proofs and their compatibility with the VHASS construction is presented is the below list;
\begin{itemize}
    \item Proof size
    \item Communication complexity
    \item Flexibility of range
    \item Assumptions and requirements 
    \item Computation complexity for prover resp. verifier
\end{itemize}

Remark that all the range proof considered aim to prove that the secret in a Pedersen commitment is in an allowed range. Thus to combine any of the range proofs with the VHASS construction, the clients needs beyond previously computed and published values also publish a Pedersen commitment. This is investigated further in section \ref{}. Another remark is that all range proofs considered have be made non interactive using the Fiat-Shamir heuristic, even if they were originally presented as interactive constructions. 

%\subsection{Square-based range proof}
\subsection{Signature-based range proof}
flexible, sets and arbitrary range proofs. 
sends $XXX$. Signature is $\mathcal{O}(n)$ or using $\sigma=\sum_{k=1}^j\sigma_ju^j$ we have $\mathcal{O}(\frac{n}{log\:n-log\:log\:n})$\\
Third party?
\subsection{Bulletproof}
bullet is $\mathcal{O}(log\:n)$. Bulletproof not general range. No third party?
 logarithmic size, linear prove and verifucation time'? 


\begin{table}[]
    \centering
    \begin{tabular}{c|c}
         bullet proof & \\
        signature  & \\
        square & \\
    \end{tabular}
    \caption{Caption}
    \label{tab:my_label}
\end{table}


\section{Additive homomorphic secret sharing with verification of both clients and severs }
\begin{algorithm}[H]
\caption{XX}
\textbf{Require:} \\
\textbf{Ensure:} 
\\\hrulefill
\begin{enumerate}
  \item 
\end{enumerate}
\label{alg: 2D esprit}
\end{algorithm}
\section{Proofs}
\begin{thm}[Correctness]
$Pr[]$
\end{thm}
\begin{proof}
To prove XXX it is sufficient to show that $\sigma= \prod_{i=1}^n \tau_i \:\bigwedge\: \prod_{i=1}^n \tau_i = \mathcal{H}(y)$. For $y$ and $\sigma$ we have the same construction as in \cite{VHASS}. Hence by construction we have:
\begin{align}
    \label{eq:y=sum(x_ij)}
    y = \sum_{j=1}^m y_j= \sum_{j=1}^m \sum_{i=1}^n \lambda_{ij}p_i(\theta_{ij}) = \sum_{i=1}^n \overbrace{ \Big (\sum_{j=1}^m \lambda_{ij}p_i(\theta_{ij}) \Big)}^{ p_i(0)} = \sum_{i=1}^n p_i(0) = \sum_{i=1}^n x_i,
\end{align}
and for $\sigma$ it holds that:
\begin{align*}
    \sigma = \prod_{j=1}^m \sigma_j = \prod_{j=1}^m g^{y_j} = g^{\sum_{j=1}^my_j} =g^y = \mathcal{H}(y)
\end{align*}
For the $\tau_i$, whose construction has been modified compared to \cite{VHASS} we have:
\begin{align*}
    &\prod_{i=1}^n \tau_i = \prod_{i=1}^n \mathds{E}(x_i,R_i)= \prod_{i=1}^n g^{x_i}h^{R_i} = g^{\sum_{i=1}^n x_i } h^{\sum_{i=1}^n R_i} \overset{\eqref{eq:y=sum(x_ij)}}{=} g^y h^{\sum_{i=1}^{n-1} R_i+R_n} = \\ 
    &= g^y h^{ \phi(N)\big\lceil \frac{\sum_{i=1}^{n-1}R_i}{\phi(N) }\big\rceil}  \overset{*}{=} g^y = \mathcal{H}(y) \quad \textit{*- since $h$ is co-prime to $N$.}
\end{align*}

\end{proof}

\section{Implementation}